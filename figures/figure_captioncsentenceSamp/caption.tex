\csentence{Sample figure title.}
      A) Biplot for PCA of log transformed assembly metrics

Biplot of the first and second principal components for the log transformed assembly metrics. Multicolored samples representing individual assemblies are identified by genus. The assembly metrics included as vectors are indicated in burgundy. 

PC1 explained 41% of the variance in the included studies. Genome map N50, percent of length of \textit{in silico} map aligned with genome maps and molecule map coverage were positively  correlated with each other and negatively correlated with molecule map labels per 100 kb. If higher label density increased the number of fragile sites within the genome this could  negatively effect the proportion of the molecules that remain long enough for molecule length filters and could impede assembly across fragile sites in genome maps. 

PC2 explained 25.6% of the variance. Primarily in PC2 molecule map coverage negatively correlated with the percent of the length of in silico maps that align to genome maps. One possible contributor to this pattern could be that the Irys System and software is optimized for Human samples and therefore we often run less material and get very high quality alignments with these supported samples.

Because many of the genomic metrics had very broad ranges with variance that increased often for higher values the genomic metrics were log transformed to compress the upper tails and stretch the lower tails.
