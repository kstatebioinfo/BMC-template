Biplot of the first and second principal components for the log transformed assembly metrics. Multicolored samples representing individual assemblies are identified by genus. The assembly metrics included as vectors are indicated in burgundy.

PC1 explained 42\% of the variance in the included studies. In PC1, FASTA N50, genome map N50, increase in FASTA N50 after super scaffolding, percent of \textit{in silico} map length aligned to genome maps all showed a strong positive correlation to each other. Genome FASTA length also showed a positive correlation with these variables.

PC2 explained 20.6\% of the variance. In this PC molecule map coverage is negatively correlated with molecule map labels per 100 kb. Labels per 100 kb values are monitored as data is collected and compared to estimated label density. Lower than expected label density can occasionally lead to further labeling reactions or other adjustments to data collection and therefore greater depth of coverage. This may be more common for unsupported samples (i.e. a species that has not previously been used to create molecule maps).

Because many of the genomic metrics had very broad ranges with variance that increased often for higher values the genomic metrics were log transformed to compress the upper tails and stretch the lower tails.